\section{Question 4}

\begin{question}
    Consider the equation $u_t+a u_x=0$ where " $a$ "is a constant. Verify that the center difference scheme (with forward Euler time discretization) has a Fourier symbol whose magnitude is $\leqslant 2$ if the CFL number is $\leqslant \sqrt{3}$. Therefore it can be stabilized by BFECC with CFL number $\leqslant \sqrt{3}$.
\end{question}

\begin{answer}
    \begin{proof}
        The center difference scheme is:
        \begin{equation}
            \begin{aligned}
                &\tfrac{U_j^{n+1} - U_j^n}{\Delta t} + a\cdot \tfrac{U_{j+1}^n - U_{j-1}^{n}}{2\Delta x}\\
                \Rightarrow &U_j^{n+1} = -\tfrac{R}{2}U_{j+1}^n + U_j^n + \tfrac{R}{2}U_{j-1}^n
            \end{aligned}
        \end{equation}
        where $R = a\cdot \tfrac{\Delta t}{\Delta x}$. If we apply the Fourier transform on it, we will have:
        \begin{equation}
            \widehat{U^{n+1}}(\xi) = (-\tfrac{R}{2}e^{i\xi} + 1 + \tfrac{R}{2}e^{-i\xi})\widehat{U^n}(\xi)
        \end{equation}
        Hence, our Fourier symboal $\rho(\xi) = -\tfrac{R}{2}e^{i\xi} + 1 + \tfrac{R}{2}e^{-i\xi} = 1-ir\sin(\xi)$ by applying the Euler's Formula. Because the CFL number is $\leq \sqrt{3}$, then we know that $\lvert R \rvert \leq \sqrt{3}$. Thus,
        \begin{equation}
            \lvert \rho(\xi) \rvert = \sqrt{1^2 + r^2\sin^2(\xi)} \leq \sqrt{1^2 + r^2 \cdot 1^2} \leq \sqrt{1+3} = 2
        \end{equation}
        Hence, this would implies $\lvert \rho_{\text{BFECC}} \rvert \leq 1$, so that is can be stabilized by BFECC with CFL number $\leq \sqrt{3}$.
    \end{proof}
\end{answer}